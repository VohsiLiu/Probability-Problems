\documentclass[10pt,a4paper]{article}
\usepackage[UTF8]{ctex}
\usepackage{fontspec}
\usepackage{geometry} 
\usepackage{amsmath}
\usepackage[shortlabels]{enumitem}
\usepackage{float}
\usepackage{graphicx}
\usepackage{subfigure}
\usepackage{epstopdf}
\usepackage{amsmath,amssymb}
\usepackage{diagbox}
\usepackage{setspace}
\usepackage{enumitem}
\usepackage[table,xcdraw]{xcolor}
\usepackage{multirow}
\DeclareSymbolFont{EulerExtension}{U}{euex}{m}{n}
\DeclareMathSymbol{\euintop}{\mathop} {EulerExtension}{"52}
\DeclareMathSymbol{\euointop}{\mathop} {EulerExtension}{"48}
\let\intop\euintop
\let\ointop\euointop

\geometry{left=3.17cm,right=3.17cm,top=2.53cm,bottom=2.54cm}
%\setmainfont{Times New Roman}
\pagestyle{plain}
\setlist[enumerate,1]{label=\textbf{\arabic*.}}
\setlist[enumerate,2]{label=(\arabic*)}

\begin{document}

    
\begin{enumerate}


    \item 设$X_1,X_2$是数学期望为$\theta$的指数分布总体$X$的容量为2的样本,设$Y=\sqrt{X_1X_2}$,试
    证明:
    $$E(\frac{4Y}{\pi})=\theta$$





    \item 设总体$X\sim N(\mu,\sigma^2),X_1,X_2,\cdots,X_n$是一个样本,$\overline{X},S^2$分别为样本均值和样本方差,
    试证
    $$E[(\overline{X}S^2)^2]=\left(\frac{\sigma^2}{n}+\mu^2\right)\left(\frac{2\sigma^4}{n-1}+\sigma^4\right)$$







    \item 设总体$X$具有概率密度
    $$f(x)=\left\{\begin{array}{ll}
        \dfrac{1}{\theta}xe^{-x/\theta}, & x>0\\
        0, & x\leq 0
    \end{array}\right.$$
    其中$\theta>0$为未知参数,$X_1,X_2,\cdots,X_n$是来自$X$的样本,$x_1,x_2,\cdots,x_n$是相应的样本观察值.
    \begin{enumerate}
        \item 求$\theta$的最大似然估计量.
        \item 求$\theta$的矩估计量.
        \item 问求得的估计量是否是无偏估计量.
    \end{enumerate}




    \item 设$X_1,X_2,\cdots,X_n$以及$Y_1,Y_2,\cdots,Y_n$为分别来自总体$N(\mu_1,\sigma^2)$与$N(\mu_2,\sigma^2)$的样本,
    且它们相互独立,$\mu_1,\mu_2,\sigma^2$均未知,试求$\mu_1,\mu_2,\sigma^2$的最大似然估计量.




    \item 为了研究一批贮存着的产品的可靠性,在产品投入贮存时, 即在时刻$t_0=0$时,随机
    地选定$n$件产品,然后在预先规定的时刻$t_1,t_2,\cdots,t_k$取出来进行检测(检测时确定已失效的
    去掉,将未失效的继续投入贮存),今得到以下的寿命试验数据:
    \renewcommand{\arraystretch}{1.4}
    \begin{table}[H]\centering
        \begin{tabular}{c|cccccc}
        \hline
        检测时刻(月)              & $t_1$     & $t_2$       & $\cdots$ & $t_k$           &                &                        \\ \hline
        区间$(t_{i-1},t_i]$    & $(0,t_1]$ & $(t_1,t_2]$ & $\cdots$ & $(t_{k-1},t_k]$ & $(t_k,\infty)$ &                        \\ \hline
        在$(t_{i-1},t_i]$的失效数 & $d_1$     & $d_2$    & $\cdots$ & $d_k$           & $s$              & $\displaystyle{\sum_{i=1}^k d_i+s=n}$ \\ \hline
        \end{tabular}
    \end{table}
    \renewcommand{\arraystretch}{1.0}
    这种数据称为区间数据.设产品寿命$T$服从指数分布,其概率密度为
    $$f(t)=\left\{\begin{array}{ll}
        \lambda e^{-\lambda t}, & t>0,\\
        0, & \mbox{其他},
    \end{array}\right.\quad \lambda>0\mbox{未知}$$
    \begin{enumerate}
        \item 试基于上述数据写出入的对数似然方程.(提示:考虑事件“$n$只产品分别在区间
        $(0,t_1],(t_1,t_2],\cdots,(t_{k-1},t_k]$失效$d_1,d_2,\cdots,d_k$只,而直至$t_k$还有$s$只未失效”的概率.)
        \item 设$d_1<n,s<n$,我们可以用数值解法求得$\lambda$的最大似然估计值,在计算机上计算是
        容易的. 特别,取检测时间是等间隔的,即取$t_i=it_1,i=1,2,\cdots,k$.验证,此时可得$\lambda$的最大似
        然估计为
        $$\hat{\lambda}=\frac{1}{t_1}\ln\left(1+\frac{n-s}{\displaystyle{\sum_{k=2}^k(i-1)d_i+sk}}\right)$$
    \end{enumerate}  





    \item 设某种电子器件的寿命(以小时计)$T$服从指数分布,概率密度为
    $$f(t)=\left\{\begin{array}{ll}
        \lambda e^{-\lambda t}, & t>0,\\
        0, & \mbox{其他},
    \end{array}\right.$$
    其中$\lambda>0$未知.从这批器件中任取$n$只在时刻$t=0$时投入独立寿命试验.试验进行到预定
    时间$T_0$结束.此时,有$k(0<k<n)$只器件失效,试求$\lambda$的最大似然估计.(提示:考虑“试验直
    至时刻$T_0$为止,有$k$只器件失效,而有$n-k$只未失效”这一事件的概率,从而写出$\lambda$的似然
    方程.)






    \item 设系统由两个独立工作的成败型元件串联而成(成败型元件只有两种状态:正常工作
    或失效).元件1、元件2的可靠性分别为$p_1,p_2$,它们均未知.随机地取$N$个系统投入试验,
    当系统中至少有一个元件失效时系统失效,现得到以下的试验数据:$n_1$——仅元件1失效的系
    统数; $n_2$——仅元件2失效的系统数;$n_{12}$——元件1,元件2至少有一个失效的系统数; $s$——未失效
    的系统数.$n_1+n_2+n_{12}+s=N$.这里$n_{12}$为隐蔽数据,也就是只知系统失效,但不能知道是由
    元件1还是元件2单独失效引起的,还是由元件1,2均失效引起的.设隐蔽与系统失效的真
    正原因独立.
    \begin{enumerate}
        \item 试写出$p_1,p_2$的似然函数.
        \item 设有系统寿命试验数据$N=20,n_1=5,n_2=3,n_{12}=1,s=11$.试求$p_1,p_2$的最大似然
        估计.(提示:$p_1$应满足方程$(p_1-1)(12p_1^2+11p_1-14)=0$.)
    \end{enumerate}





    \item \begin{enumerate}
        \item 设总体$X$具有分布律
        \renewcommand{\arraystretch}{1.3}
        \begin{table}[H]\centering
            \begin{tabular}{c|ccc}
            $X$   & 1        & 2        & 3           \\ \hline
            $p_k$ & $\theta$ & $\theta$ & $1-2\theta$
            \end{tabular}
        \end{table}
        \renewcommand{\arraystretch}{1.0}
        $\theta>0$未知,今有样本
        $$\begin{array}{cccccccccccccccc}
            1 & 1 & 1 & 3 & 2 & 1 & 3 & 2 & 2 & 1 & 2 & 2 & 3 & 1 & 1 & 2 
        \end{array}$$
        试求$\theta$的最大似然估计值和矩估计值.
        \item 设总体$X$服从$\Gamma$分布.其概率密度为
        $$f(x)=\left\{\begin{array}{ll}
            \dfrac{1}{\beta^\alpha\Gamma(\alpha)}x^{\alpha-1}e^{-x/\beta}, & x>0\\
            0, & \mbox{其他}
        \end{array}\right.$$
        其形状参数$\alpha>0$为已知,尺度参数$\beta>0$未知.今有样本值$x_1,x_2,\cdots,x_n$,求$\beta$的最大似然估
        计值.
    \end{enumerate}





    \item \begin{enumerate}
        \item 设$Z=\ln X\sim N(\mu,\sigma^2)$,即$X$服从对数正态分布,验证$E(X)=\exp (\mu+\dfrac{1}{2}\sigma^2)$.
        \item 设自(1)中总体$X$中取一容量为$n$的样本$x_1,x_2,\cdots,x_n$,求$E(X)$的最大似然估计.
        此处设$\mu,\sigma^2$均为未知.
        \item 已知在文学家萧伯纳的《An Intelligent Woman's Guide To Socialism》一书中,一个句
        子的单词数近似地服从对数正态分布,设$\mu,\sigma^2$为未知.今自该书中随机地取20个句子.这
        些句子中的单词数分别为
        $$\begin{array}{cccccccccc}
            52 & 24 & 15 & 67 & 15 & 22 & 63 & 26 & 16 & 32\\
            7 & 33 & 28 & 14 & 7 & 29 & 10 & 6 & 59 & 30 
        \end{array}$$
        问这本书中,一个句子单词数均值的最大似然估计值等于多少?
    \end{enumerate}






    \item 考虑进行定数截尾寿命试验,假设将随机抽取的n 件产品在时间$t=0$时同时投入试
    验试验进行到$m$件$(m<n)$产品失效时停止,$m$件失效产品的失效时间分别为
    $$0\leq t_1 \leq t_2 \leq \cdots \leq t_m$$
    $t_m$是第$m$件产品的失效时间.设产品的寿命分布为韦布尔分布,其概率密度为
    $$f(x)=\left\{\begin{array}{ll}
        \dfrac{\beta}{\eta ^\beta}x^{\beta-1}e^{{-(\frac{x}{\eta})}^\beta}, & x>0\\
        0, & \mbox{其他}
    \end{array}\right.$$
    其中参数$\beta$已知.求参数$\eta$的最大似然估计.





    \item 设某大城市郊区的一条林荫道两旁开设了许多小商店,这些商店的开设延续时间(以
    月计)是一个随机变量,现随机地取30家商店,将它们的延续时间按自小到大排序,选其中前
    8家商店.它们的延续时间分别是
    $$\begin{array}{cccccccc}
        3.2 & 3.9 & 5.9 & 6.5 & 16.5 & 20.3 & 40.4 & 50.9
    \end{array}$$
    假设商店开设延续时间的长度是韦布尔随机变量.其概率密度为
    $$f(x)=\left\{\begin{array}{ll}
        \dfrac{\beta}{\eta ^\beta}x^{\beta-1}e^{{-(\frac{x}{\eta})}^\beta}, & x>0\\
        0, & \mbox{其他}
    \end{array}\right.$$
    其中,$\beta=0.8$.
    \begin{enumerate}
        \item 试用上题结果,写出$\eta$的最大似然估计.
        \item 按(1)的结果求商店开设延续时间至少为2年的概率的估计.
    \end{enumerate}






    \item 设分别自总体$N(\mu_1,\sigma^2)$和$N(\mu_2,\sigma^2)$中抽取容量$n_1,n_2$的两独立样本,其样本方差分
    别为$S_1^2,S_2^2$.试证.对于任意常数$a,b(a+b=1),Z=aS_1^2+bS_2^2$都是$\sigma^2$的无偏估计,并确定常
    数$a,b$,从使$D(Z)$达到最小.





    \item 设总体$X\sim N(\mu,\sigma^2)$,$X_1,X_2,\cdots,X_n$是来自$X$的样本.已知样本方差$S^2=\dfrac{1}{n-1}\displaystyle{\sum ^n_{i=1} (X_i-\overline{X})^2}$是$\sigma^2$的无
    偏估计.验证样本标准差$S$不是标准差$\sigma$的无偏估计.(提示:
    记$Y=\dfrac{(n-1)S^2}{\sigma^2}$,则$Y\sim \chi^2(n-1)$,而$S=\dfrac{\sigma}{\sqrt{n-1}}\sqrt{Y}$是$Y$的函数,利用$\chi^2(n-1)$的概率密度
    可得$E(S)=\dfrac{1}{(\sqrt{2})^n}\sqrt{\dfrac{2}{n-1}}\dfrac{\Gamma(n/2)\sigma}{\Gamma((n-1)/2)}\neq \sigma$.)
    





    \item 设总体$X$服从指数分布,其概率密度为
    $$f(x)=\left\{\begin{array}{ll}
        \dfrac{1}{\theta}x^{-x/\theta}, &  x>0\\
        0, & \mbox{其他}
    \end{array}\right.$$
    $\theta>0$未知.从总体中抽取一容量为$n$的样本$X_1,X_2,\cdots,X_n$.
    \begin{enumerate}
        \item 证明$\dfrac{2n\overline{X}}{\theta}\sim \chi^2(2n)$.
        \item 求$\theta$的置信水平为$1-\alpha$的单侧置信下限.
        \item 某种元件的寿命(以h计)服从上述指数分布,现从中抽得一容量$n=16$的样本,测
        得样本均值为5010$\, $h, 试求元件的平均寿命的置信水平为0.90的单侧置信下限.
    \end{enumerate}




    \item 设总体$X\sim U(0,\theta)$,$X_1,X_2,\cdots,X_n$是来自$X$的样本.
    \begin{enumerate}
        \item 验证$Y=\max\{X_1,X_2,\cdots,X_n\}$的分布函数为
        $$F_Y(y)=\left\{\begin{array}{ll}
            0, & y>0\\
            y^n/\theta^n, & 0\leq y <\theta\\
            1, &  y\geq \theta
        \end{array}\right.$$
        \item 验证$U=Y/\theta$的概率密度为
        $$f_U(u)=\left\{\begin{array}{ll}
            nu^{n-1}, & 0\leq u \leq 1\\
            0, & \mbox{其他}
        \end{array}\right.$$
        \item 给定正数$\alpha,0<\alpha<1$,求$U$的分布的上$\alpha/2$分位点$h_{\alpha/2}$以及上$1-\alpha/2$分位点$h_{1-\alpha/2}$.
        \item 利用(2),(3) 求参数$\theta$的置信水平为$1-\alpha$的置信区间.
        \item 设某人上班的等车时间$X\sim U(0,\theta)$,$\theta$未知.现在有样本$x_1=4.2,x_2=3.5,x_3=1.7,x_4=1.2,x_5=2.4$,
        求$\theta$的置信水平为0.95的置信区间.
    \end{enumerate}





    \item 设总体$X$服从指数分布,概率密度为
    $$f(x)=\left\{\begin{array}{ll}
        \dfrac{1}{\theta}e^{-x/\theta}, & x>0\\
        0, & \mbox{其他}
    \end{array}\right.\quad \theta>0$$
    设$X_1,X_2,\cdots,X_n$是来自$X$的一个样本.试取第14题中当$\theta=\theta_0$时的统计量$\chi^2=\dfrac{2n\overline{X}}{\theta_0}$作为检
    验统计量,检验假设$H_0:\theta=\theta_0,H_1:\theta\neq \theta_0$.取显著性水平为$\alpha$(注意:$E(\overline{X})=\theta$)

    设某种电子元件的寿命(以小时计)服从均值为$\theta$的指数分布,随机取12只元件测得它
    们的寿命分别为
    $$\begin{array}{cccccccccccc}
        340 & 430 & 560 & 920 & 1380 & 1520 & 1660 & 1770 & 2100 & 2320 & 2350 & 2650
    \end{array}$$
    试取显著性水平$\alpha=0.05$,检验假设$H_0:\theta=1450,H_1:\theta\neq 1450$.





    \item 经过十一年的试验,达尔文于1876年得到15对玉米样品的数据如下表,每对作物除
    授粉方式不同外,其他条件都是相同的.试用逐对比较法检验不同授粉方式对玉米高度是否
    有显著的影响$(\alpha=0.05)$.问应增设什么条件才能用逐对比较法进行检验?
    \renewcommand{\arraystretch}{2.5}
    \begin{table}[H]\centering
        \begin{tabular}{c|c|c|c|c|c|c|c|c}
        \hline
        授粉方式             & 1                & 2                & 3                & 4  & 5                & 6                & 7                & 8                \\ \hline
        异株授粉的作物高度($x_i$) & $23\dfrac{1}{8}$ & 12               & $20\dfrac{3}{8}$ & 22 & $19\dfrac{1}{8}$ & $21\dfrac{4}{8}$ & $22\dfrac{1}{8}$ & $20\dfrac{3}{8}$ \\ \hline
        同株授粉的作物高度($y_i$) & $27\dfrac{3}{8}$ & 21               & 20               & 20 & $19\dfrac{3}{8}$ & $18\dfrac{5}{8}$ & $18\dfrac{5}{8}$ & $15\dfrac{2}{8}$ \\ \hline
        授粉方式             & 9                & 10               & 11               & 12 & 13               & 14               & 15               &                  \\ \hline
        异株授粉的作物高度($x_i$) & $18\dfrac{2}{8}$ & $21\dfrac{5}{8}$ & $23\dfrac{2}{8}$ & 21 & $22\dfrac{1}{8}$ & 23               & 12               &                  \\ \hline
        同株授粉的作物高度($y_i$) & $16\dfrac{4}{8}$ & 18               & $16\dfrac{2}{8}$ & 18 & $12\dfrac{6}{8}$ & $15\dfrac{4}{8}$ & 18               &                  \\ \hline
        \end{tabular}
    \end{table}
    \renewcommand{\arraystretch}{1.0}





    \item 一内科医生声称,如果病人每天傍晚聆听一种特殊的轻音乐会降低血压(舒张压,
    以mmHg计).今选取了10个病人在试验之前和试验之后分别测量了血压,得到以下的
    数据:
    \renewcommand{\arraystretch}{1.3}
    \begin{table}[H]\centering
        \begin{tabular}{c|c|c|c|c|c|c|c|c|c|c}
        \hline
        病人          & 1  & 2  & 3  & 4  & 5  & 6  & 7  & 8  & 9  & 10 \\ \hline
        试验之前($x_i$) & 86 & 92 & 95 & 84 & 80 & 78 & 98 & 95 & 94 & 96 \\ \hline
        试验之后($y_i$) & 84 & 83 & 81 & 78 & 82 & 74 & 86 & 85 & 80 & 82 \\ \hline
        \end{tabular}
    \end{table}
    \renewcommand{\arraystretch}{1.0}
    设$D_i=X_i-Y_i(i=1,2,\cdots,10)$为来自正态总体$N(\mu_D,\sigma_D^2)$的样本,$\mu_D,\sigma^2_D$均未知.试检验是否
    可以认为医生的意见是对的(取$\alpha=0.05$).




    \item 以下是各种颜色的汽车的销售情况:
    \renewcommand{\arraystretch}{1.3}
    \begin{table}[H]\centering
        \begin{tabular}{c|c|c|c|c|c}
        \hline
        颜色  & 红  & 黄  & 蓝  & 绿  & 棕  \\ \hline
        车辆数 & 40 & 64 & 46 & 36 & 14 \\ \hline
        \end{tabular}
    \end{table}
    \renewcommand{\arraystretch}{1.0}
    试检验顾客对这些颜色是否有偏爱,即检验销售情况是否是均匀的(取$\alpha=0.05$).






    \item 某种闪光灯,每盏灯含4个电池,随机地取150盏灯,经检测得到以下的数据:
    \renewcommand{\arraystretch}{1.3}
    \begin{table}[H]\centering
        \begin{tabular}{c|c|c|c|c|c}
        \hline
        一盏灯损坏的电池数$x$ & 0  & 1  & 2  & 3  & 4  \\ \hline
        灯的盏数         & 26 & 51 & 47 & 16 & 10 \\ \hline
        \end{tabular}
    \end{table}
    \renewcommand{\arraystretch}{1.0}
    试取$\alpha=0.05$检验一盏灯损坏的电池数$X\sim b(4,\theta)$($\theta$未知).





    \item 临界闪烁频率(cff)是人眼对于闪烁光源能够分辨出它在闪烁的最高频率(以Hz
    计).超过cff的频率,即使光源实际是在闪烁的,而人看起来是连续的(不闪烁的).一项研究
    旨在判定cff的均值是否与人眼的虹膜颜色有关.所得数据如下:
    \renewcommand{\arraystretch}{1.3}
    \begin{table}[H]\centering
        \begin{tabular}{ccccccc}
        \hline
        虹膜颜色                 & \multicolumn{2}{c}{棕色}                              & \multicolumn{2}{c}{绿色}                          & \multicolumn{2}{c}{蓝色}                         \\ \hline
                             & 26.8                     & 26.3                     & 26.4                     & 29.1                 & 25.7                     & 29.4                 \\
                             & 27.9                     & 24.8                     & 24.2                     &                      & 27.2                     & 28.3                 \\
        \multicolumn{1}{l}{} & \multicolumn{1}{l}{23.7} & \multicolumn{1}{l}{25.7} & \multicolumn{1}{l}{28.0} & \multicolumn{1}{l}{} & \multicolumn{1}{l}{29.9} & \multicolumn{1}{l}{} \\
                             & 25.0                     & 24.5                     & 26.9                     &                      & 28.5                     &                      \\ \hline
        \end{tabular}
    \end{table}
    \renewcommand{\arraystretch}{1.0}
    试在显著性水平0.05下,检验各种虹膜颜色相应的cff的均值有无显著的差异.设各个总体
    服从正态分布,且方差相等,不同颜色下的样本之间相互独立.




    \item 面列出了挪威人自$1938 \sim 1947$年间年人均脂肪消耗量与患动脉粥样硬化症而死
    亡的死亡率之间相关的一组数据.
    \renewcommand{\arraystretch}{1.3}
    \begin{table}[H]\centering
        \begin{tabular}{c|c|c|c|c|c|c|c|c|c|c}
        \hline
        年份                   & 1938 & 1939 & 1940 & 1941 & 1942 & 1943 & 1944 & 1945 & 1946 & 1947 \\ \hline
        脂肪消耗量$x$(千克/人年)      & 14.4 & 16.0 & 11.6 & 11.0 & 10.0 & 9.6  & 9.2  & 10.4 & 11.4 & 12.5 \\ \hline
        死亡率$y$(1/($10^5$人年)) & 29.1 & 29.7 & 29.2 & 26.0 & 24.0 & 23.1 & 23.0 & 23.1 & 25.2 & 26.1 \\ \hline
        \end{tabular}
    \end{table}
    \renewcommand{\arraystretch}{1.0}
    设对于给定的$x,Y$为正态变量,且方差与$x$无关.
    \begin{enumerate}
        \item 求回归直线方程$y=a+bx$.
        \item 在显著性水平$\alpha=0.05$下检验假设$H_0:b=0,H_1:b\neq 0$.
        \item 求$\hat{y}|_{x=13}$.
        \item 求$x=13$处$\mu(x)$置信水平为0.95的置信区间.
        \item 求$x=13$处$Y$的新观察值$Y_0$的置信水平为0.95的预测区间.
    \end{enumerate}




    \item 下面给出$1924 \sim 1992$年奥林匹克运动会女子100米仰泳的最佳成绩(以s计),(其
    中1940年及1944年未举行奥运会)
    \renewcommand{\arraystretch}{1.3}
    \begin{table}[H]\centering
        \begin{tabular}{c|cccccccc}
        \hline
        年份 & 1924 & 1928 & 1932 & 1936 & 1948 & 1952 & 1956 & 1960 \\ \hline
        成绩 & 83.2 & 82.2 & 79.4 & 78.9 & 74.4 & 74.3 & 72.9 & 69.3 \\ \hline
        年份 & 1964 & 1968 & 1972 & 1976 & 1980 & 1984 & 1988 & 1992 \\ \hline
        成绩 & 67.7 & 66.2 & 65.8 & 61.8 & 60.9 & 62.6 & 60.9 & 60.7 \\ \hline
        \end{tabular}
    \end{table}
    \renewcommand{\arraystretch}{1.0}
    \begin{enumerate}
        \item 画出散点图.
        \item 求成绩关于年份的线性回归方程.
        \item 检验回归效果是否显著(取$\alpha=0.05$).
    \end{enumerate}











  

\end{enumerate}
\end{document}