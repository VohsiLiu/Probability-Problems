\documentclass[10pt,a4paper]{article}
\usepackage[UTF8]{ctex}
\usepackage{fontspec}
\usepackage{geometry} 
\usepackage{amsmath}
\usepackage[shortlabels]{enumitem}
\usepackage{float}
\usepackage{graphicx}
\usepackage{subfigure}
\usepackage{epstopdf}
\usepackage{amsmath,amssymb}
\usepackage{diagbox}
\usepackage{setspace}
\usepackage{enumitem}
\usepackage[table,xcdraw]{xcolor}
\usepackage{multirow}
\DeclareSymbolFont{EulerExtension}{U}{euex}{m}{n}
\DeclareMathSymbol{\euintop}{\mathop} {EulerExtension}{"52}
\DeclareMathSymbol{\euointop}{\mathop} {EulerExtension}{"48}
\let\intop\euintop
\let\ointop\euointop

\geometry{left=3.17cm,right=3.17cm,top=2.53cm,bottom=2.54cm}
%\setmainfont{Times New Roman}
\pagestyle{plain}
\setlist[enumerate,1]{label=\textbf{\arabic*.}}
\setlist[enumerate,2]{label=(\arabic*)}

\begin{document}
    以下约定各个习题均符合涉及的方差分析模型或回归分析模型所要求的条件.
\begin{enumerate}


    \item 今有某种型号的电池三批,它们分别是$A,B,C$三个工广所生产的.为评比其质拔,各
    随机抽取5只电池为样品,经试验得其寿命(h)如下:
    \renewcommand{\arraystretch}{1.3}
    \begin{table}[H]\centering
        \begin{tabular}{cccccc}
        \hline
        \multicolumn{2}{c}{$A$} & \multicolumn{2}{c}{$B$} & \multicolumn{2}{c}{$C$} \\ \hline
        40         & 42         & 26         & 28         & 39         & 50         \\
        48         & 45         & 34         & 32         & 40         & 50         \\
        \multicolumn{2}{c}{38}  & \multicolumn{2}{c}{30}  & \multicolumn{2}{c}{43}  \\ \hline
        \end{tabular}
    \end{table}
    \renewcommand{\arraystretch}{1.0}
    试在显著性水平0.05下检验电池的平均寿命有无显著的差异.若差异是显著的,试求均值差
    $\mu_A-\mu_B,\mu_A-\mu_C$和$\mu_B-\mu_C$的置信水平为95\%的置信区间.



    
    \item 为了寻找飞机控制板上仪器表的最佳布置,试验了三个方案.观察领航员在紧急情况
    的反应时间(以$1/10$秒计),随机地选择28名领航员.得到他们对于不同的布置方案的反应时
    间如下:
    \renewcommand{\arraystretch}{1.3}
    \begin{table}[H]\centering
        \begin{tabular}{ccccccccccccc}
        \hline
        方案\uppercase\expandafter{\romannumeral1} & 14 & 13 & 9 & 15 & 11 & 13 & 14 & 11 &    &   &    &   \\ \hline
        方案\uppercase\expandafter{\romannumeral2} & 10 & 12 & 7 & 11 & 8  & 12 & 9  & 10 & 13 & 9 & 10 & 9 \\ \hline
        方案\uppercase\expandafter{\romannumeral3} & 11 & 5  & 9 & 10 & 6  & 8  & 8  & 7  &    &   &    &   \\ \hline
        \end{tabular}
    \end{table}
    \renewcommand{\arraystretch}{1.0}
    试在显著性水平0.05下检验各个方案的反应时间有无显著差异.若有差异.试求$\mu_1-\mu_2$,
    $\mu_1-\mu_3,\mu_2-\mu_3$的置信水平为0.95的置信区间.





    \item 某防治站对4个林场的松毛虫密度进行调查,每个林场调查5块地得资料如下表:
    \renewcommand{\arraystretch}{1.3}
    \begin{table}[H]\centering
        \begin{tabular}{cccccc}
        \hline
        地点    & \multicolumn{5}{c}{松毛虫密度(头/标准地)} \\ \hline
        $A_1$ & 192  & 189  & 176  & 185  & 190  \\
        $A_2$ & 190  & 201  & 187  & 196  & 200  \\
        $A_3$ & 188  & 179  & 191  & 183  & 194  \\
        $A_4$ & 187  & 180  & 188  & 175  & 182  \\ \hline
        \end{tabular}
        \end{table}
    判断4个林场松毛虫密度有无显著差异.取显著性水平$\alpha=0.05$.
    \renewcommand{\arraystretch}{1.0}




    \item 一试验用来比较4种不同药品解除外科手术后疼痛的延续时间(h),结果如下表:
    \renewcommand{\arraystretch}{1.3}
    \begin{table}[H]\centering
        \begin{tabular}{cccccc}
        \hline
        药品  & \multicolumn{5}{c}{时间长度(h)} \\ \hline
        $A$ & 8   & 6   & 4   & 2   &     \\
        $B$ & 6   & 6   & 4   & 4   &     \\
        $C$ & 8   & 10  & 10  & 10  & 12  \\
        $D$ & 4   & 4   & 2   &     &     \\ \hline
        \end{tabular}
    \end{table}
    \renewcommand{\arraystretch}{1.0}
    试在显著性水平$\alpha=0.05$下检验各种药品对解除疼痛的延续时间有无显著差异.




    \item 将抗生素注入人体会产生抗生素与血浆蛋白质结合的现象,以致减少了药效.下表列
    出5种常用的抗生素注入牛的体内时,抗生索与血浆蛋白质结合的百分比.
    \renewcommand{\arraystretch}{1.3}
    \begin{table}[H]\centering
        \begin{tabular}{ccccc}
        \hline
        青霉素  & 四环素  & 链霉素  & 红霉素  & 氯霉素  \\ \hline
        29.6 & 27.3 & 5.8  & 21.6 & 29.2 \\
        24.3 & 32.6 & 6.2  & 17.4 & 32.8 \\
        28.5 & 30.8 & 11.0 & 18.3 & 25.0 \\
        32.0 & 34.8 & 8.3  & 19.0 & 24.2 \\ \hline
        \end{tabular}
    \end{table}
    \renewcommand{\arraystretch}{1.0}
    试在显著性水平$\alpha=0.05$检验这些百分比的均值有无显著的差异.



    \item 下表给出某种化工过程在三种浓度、四种温度水平下得率的数据:
    \renewcommand{\arraystretch}{1.8}
    \begin{table}[H]\centering
    \begin{tabular}{cc|cccc}
    \hline
                                                                          &     & \multicolumn{4}{c}{温度(因素$B$)}                                                                                                \\ \cline{3-6} 
                                                                          &     & 10$^{\circ}$C                                 & 24$^{\circ}$C                                 & 38$^{\circ}$C                                 & 52$^{\circ}$C            \\ \hline
    \multirow{3}{*}{\begin{tabular}[c]{@{}c@{}}浓度\\ (因素$A$)\end{tabular}} & 2\% & \multicolumn{1}{c|}{$14 \quad 10$} & \multicolumn{1}{c|}{$11 \quad 11$} & \multicolumn{1}{c|}{$13 \quad 9$}  & $10 \quad 12$ \\ \cline{3-6} 
                                                                          & 4\% & \multicolumn{1}{c|}{$9 \quad 7$}   & \multicolumn{1}{c|}{$10 \quad 8$}  & \multicolumn{1}{c|}{$7 \quad 11$}  & $6 \quad 10$  \\ \cline{3-6} 
                                                                          & 6\% & \multicolumn{1}{c|}{$5 \quad 11$}  & \multicolumn{1}{c|}{$13\quad 14$}  & \multicolumn{1}{c|}{$12 \quad 13$} & $14 \quad 10$ \\ \hline
    \end{tabular}
    \end{table}
    \renewcommand{\arraystretch}{1.0}
    试在显茗性水平$\alpha=0.05$下检验:在不同浓度下得率的均值是否有显著差异,在不同温度下
    得率的均值是否有显著差异.交互作用的效应是否显著.




    \item 为了研究某种金属管防腐蚀的功能,考虑了4种不同的涂料涂层.将金属管埋设在3
    种不同性质的土壤中,经历了一定时间,消得金属管腐蚀的最大深度如下表所示(以mm计):
    \renewcommand{\arraystretch}{1.3}
    \begin{table}[H]\centering
    \begin{tabular}{c|ccc}
    \hline
                                                                          & \multicolumn{3}{c}{土壤类型(因素$B$)} \\ \hline
    \multirow{5}{*}{\begin{tabular}[c]{@{}c@{}}涂层\\ (因素$A$)\end{tabular}} & 1         & 2        & 3        \\ \cline{2-4} 
                                                                          & 1.63      & 1.35     & 1.27     \\
                                                                          & 1.34      & 1.30     & 1.22     \\
                                                                          & 1.19      & 1.14     & 1.27     \\
                                                                          & 1.30      & 1.09     & 1.32     \\ \cline{1-4} 
    \end{tabular}
    \end{table}
    \renewcommand{\arraystretch}{1.0}
    试取显著性水平$\alpha=0.05$检验在不同涂层下腐蚀的最大深度的平均值有无显著差异.在不同
    土壤下腐蚀的最大深度的平均值有无显著差异.设两因素间没有交互作用效应.




    \item 下表数据是退火温度($^{\circ}\mathrm{C}$)对黄铜延性$Y$效应的试验结果.$Y$是以延长度计算的.
    \renewcommand{\arraystretch}{1.3}
    \begin{table}[H]\centering
        \begin{tabular}{c|cccccc}
        $x(^{\circ}\mathrm{C})$   & 300 & 400 & 500 & 600 & 700 & 800 \\ \hline
        $y(\%)$ & 40  & 50  & 55  & 60  & 67  & 70 
        \end{tabular}
    \end{table}
    \renewcommand{\arraystretch}{1.0}
    画出散点图并求$Y$对于$x$的线性回归方程。











  

\end{enumerate}
\end{document}