\documentclass[10pt,a4paper]{article}
\usepackage[UTF8]{ctex}
\usepackage{fontspec}
\usepackage{geometry} 
\usepackage{amsmath}
\usepackage[shortlabels]{enumitem}
\usepackage{float}
\usepackage{graphicx}
\usepackage{subfigure}
\usepackage{epstopdf}
\usepackage{amsmath,amssymb}
\usepackage{diagbox}
\usepackage{setspace}
\usepackage{enumitem}
\DeclareSymbolFont{EulerExtension}{U}{euex}{m}{n}
\DeclareMathSymbol{\euintop}{\mathop} {EulerExtension}{"52}
\DeclareMathSymbol{\euointop}{\mathop} {EulerExtension}{"48}
\let\intop\euintop
\let\ointop\euointop

\geometry{left=3.17cm,right=3.17cm,top=2.53cm,bottom=2.54cm}
%\setmainfont{Times New Roman}
\pagestyle{plain}
\setlist[enumerate,1]{label=\textbf{\arabic*.}}
\setlist[enumerate,2]{label=(\arabic*)}

\begin{document}

\begin{enumerate}



    \item 在总体$N(52,6.3^2)$中随机抽取一容量为36的样本,求样本均值$\overline{X}$落在50.8到53.8之间的概率。
    \clearpage


    \item  在总体$N(12,4)$中随机抽一容量为5的样本$X_1,X_2,X_3,X_4,X_5$.
    \begin{enumerate}
        \item 求样本均值与总体均值之差的绝对值大于1的概率.
        \item 求概率$P\{\max\{X_1,X_2,X_3,X_4,X_5\}>15\},P\{\min \{X_1,X_2,X_3,X_4,X_5\}<10\}$.
    \end{enumerate}
    \clearpage


    \item 求总体$N(20,3)$的容量分别为10,15的两独立样本均值差的绝对值大于0.3的概率.
    \clearpage

    \item \begin{enumerate}
        \item 设样本$X_1,X_2,\cdots,X_6$来自总体$N(0,1),Y=(X_1+X_2+X_3)^2+(X_4+X_5+X_6)^2$,
        试确定常数$C$使$CY$服从$\chi ^2$分布.
        \item 设样本$X_1,X_2,\cdots,X_5$来自总体$N(0,1),Y=\dfrac{C(X_1+X_2)}{(X_3^2+X_4^2+X_5^2)^{1/2}}$,试确定
        常数$C$使$Y$服从$t$分布.
        \item 已知$X\sim t(n)$,求证$X^2\sim F(1,n)$.
    \end{enumerate}   
    \clearpage



    \item \begin{enumerate}
        \item 已知某种能力测试的得分服从正态分布$N(\mu,\sigma^2)$.随机取10个人参与这一测试.
        求他们得分的联合概率密度,并求这10个人得分的平均值小于$\mu$的概率.
        \item 在(1)中设$\mu=62,\sigma^2=25$,若得分超过70就能得奖,求至少有一人得奖的概率.
    \end{enumerate}
    \clearpage


    \item 设总体$X\sim b(1,p),X_1,X_2,\cdots,X_n$是来自$X$的样本
    \begin{enumerate}
        \item 求$(X_1,X_2,\cdots,X_n)$的分布律.
        \item 求$\displaystyle{\sum_{i=1}^n X_i}$的分布律.
        \item 求$E(\overline{X}),D(\overline{X}),E(S^2)$.
    \end{enumerate}
    \clearpage

    \item 设总体$X\sim \chi^2(n),X_1,X_2,\cdots,X_{10}$是来自$X$的样本,求$E(\overline{X}),D(\overline{X}),E(S^2)$.
    \clearpage

    \item 设总体$X\sim N(\mu,\sigma^2),X_1,X_2,\cdots,X_{10}$是来自$X$的样本.
    \begin{enumerate}
        \item 写出$X_1,X_2,\cdots,X_{10}$的联合概率密度.
        \item 写出$\overline{X}$的概率密度.
    \end{enumerate}
    \clearpage


    \item 设在总体$N(\mu,\sigma^2)$中抽得一容量为16的样本,这里$\mu,\sigma^2$均未知.
    \begin{enumerate}
        \item 求$P\{S^2/\sigma^2\leq 2.041\}$,其中$S^2$为样本方差.
        \item 求$D(S^2)$.
    \end{enumerate}
    \clearpage


    \item 下面列出了30个美国NBA球员的体重(以磅计,$1\mbox{磅}=0.454$kg) 数据.这些数据是
    从美国NBA球队$1990-1991$赛季的花名册中抽样得到的.
    $$\begin{array}{cccccccccc}
        225 & 232 & 232 & 245 & 235 & 245 & 270 & 225 & 240 & 240 \\
        217 & 195 & 225 & 185 & 200 & 220 & 200 & 210 & 271 & 240 \\
        220 & 230 & 215 & 252 & 225 & 220 & 206 & 185 & 227 & 236
    \end{array}$$
    \begin{enumerate}
        \item 画出这些数据的频率直方图(提示:最大和最小观察值分别为271和185,区间
        $[184.5,271.5]$包含所有数据,将整个区间分为5等份,为计算方便.将区间调整为$(179.5,279.5)$.
        \item 作出这些数据的箱线图.
    \end{enumerate}
    \clearpage

    

    \item {\heiti 截尾数据}$\quad$ 设数据集包含$n$个数据,将这些数据自小到大排序为
    $$x_{(1)}\leq x_{(2)} \leq \cdots \leq x_{(n)}$$
    删去$100\alpha \%$个数值小的数,同时删去$100\alpha \%$个数值大的数,将留下的数据取算术平均,记为
    $\overline{x}_\alpha$,即
    $$\overline{x}_\alpha=\frac{x_{([n\alpha]+1)}+\cdots+x_{(n-[n\alpha])}}{n-2[n\alpha]}$$
    其中$[n\alpha]$是小于或等于$n\alpha$的最大整数(一般取$\alpha$为$0.1\sim 0.2$).$\overline{x}_\alpha$
    称为$100\alpha\%$截尾均值。例如对于第10题中的数据,取$\alpha=0.1$,则有$[n\alpha]=[30\times 0.1]=3$,得
    $100\times 0.1\%$截尾均值
    $$\overline{x}_\alpha=\frac{200+200+\cdots+245+245}{30-6}=225.4167$$
    \par 若数据来自某一总体的样本,则$\overline{x}_\alpha$是一个统计量. $\overline{x}_\alpha$不受样本的极端值的影响.截尾均
    值在实际应用问题中是常会用到的.
    \par 试求第10题的30个数据的$\alpha=0.2$的截尾均值.
    

    


  

\end{enumerate}
\end{document}