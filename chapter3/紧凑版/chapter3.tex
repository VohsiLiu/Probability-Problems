\documentclass[10pt,a4paper]{article}
\usepackage[UTF8]{ctex}
\usepackage{fontspec}
\usepackage{geometry} 
\usepackage{amsmath}
\usepackage[shortlabels]{enumitem}
\usepackage{float}
\usepackage{graphicx}
\usepackage{subfigure}
\usepackage{epstopdf}
\usepackage{amsmath,amssymb}
\usepackage{diagbox}
\DeclareSymbolFont{EulerExtension}{U}{euex}{m}{n}
\DeclareMathSymbol{\euintop}{\mathop} {EulerExtension}{"52}
\DeclareMathSymbol{\euointop}{\mathop} {EulerExtension}{"48}
\let\intop\euintop
\let\ointop\euointop

\geometry{left=3.17cm,right=3.17cm,top=2.53cm,bottom=2.54cm}
%\setmainfont{Times New Roman}
\pagestyle{plain}
\setlist[enumerate,1]{label=\textbf{\arabic*.}}
\setlist[enumerate,2]{label=(\arabic*)}

\begin{document}

\begin{enumerate}




    \item 在一箱子中装有12只开关,其中2只是次品,在其中取两次.每次任取一只,考虑两
    种试验:(1)放回抽样;(2)不放回抽样.我们定义随机变量$X,Y$如下:
    $$X=\left\{\begin{array}{ll}
        0, & \mbox{若第一次取出的是正品}\\
        1, & \mbox{若第一次取出的是次品}
    \end{array}\right.\\
    Y=\left\{\begin{array}{ll}
        0, & \mbox{若第二次取出的是正品}\\
        1, & \mbox{若第二次取出的是次品}\\
    \end{array}\right.$$
    试分别就(1)、(2)两种情况,写出$X$和$Y$的联合分布律.


    \item \begin{enumerate}
        \item 盒子里装有3只黑球、2只红球、2只白球,在其中任取4只球.以$X$表示取到黑
        球的只数,以$Y$表示取到红球的只数.求$X$和$Y$的联合分布律.
        \item 在(1)中求$P\{X>Y\},P\{Y=2X\},P\{X+Y=3\},P\{X<3-Y\}$.
    \end{enumerate}


    \item 设随机变量$(X,Y)$的概率密度为
    $$f(x,y)=\left\{\begin{array}{ll}
        k(6-x-y), & 0<x<2,2<y<4\\
        0, & \mbox{其他}
    \end{array}\right.$$
    \begin{enumerate}
        \item 确定常数$k$.
        \item 求$P\{X<1,Y<3\}$.
        \item 求$P\{X<1.5\}$.
        \item 求$P\{X+Y\leq 4\}$.
    \end{enumerate}


    \item 设$X,Y$都是非负的连续型随机变量,它们相互独立.
    \begin{enumerate}
        \item 证明$$P\{X<Y\}=\int _0^\infty F_X(x)f_Y(x)\mathrm{d}x$$
        其中$F_X(x)$是$X$的分布函数,$f_Y(y)$是$Y$的概率密度.
        \item 设$X,Y$相互独立.其概率密度分别为
        $$f_X(x)=\left\{\begin{array}{ll}
            \lambda_1 e^{-\lambda_1 x}, & x>0,\\
            0, & \mbox{其他},
        \end{array}\right.\\
        f_Y(y)=\left\{\begin{array}{ll}
            \lambda_2 e^{-\lambda_2 y}, & y>0,\\
            0, & \mbox{其他},
        \end{array}\right.$$
        求$P\{X<Y\}$.
    \end{enumerate}


    \item 设随机变量$(X,Y)$具有分布函数
    $$F(x,y)=\left\{\begin{array}{ll}
        1-e^{-x}-e^{-y}+e^{-x-y}, & x>0,y>0\\
        0, & \mbox{其他}
    \end{array}\right.$$
    求边缘分布函数.


    \item 将一枚硬币掷3次,以$X$表示前2次中出现$H$的次数,以$Y$表示3次中出现$H$的次
    数.求$X,Y$的联合分布律以及$(X,Y)$的边缘分布律.


    \item 设二维随机变量$(X,Y)$的概率密度为
    $$f(x,y)=\left\{\begin{array}{ll}
        4.8y(2-x), & 0\leq x \leq 1,0\leq y \leq x,\\
        0, & \mbox{其他},
    \end{array}\right.$$
    求边缘概率密度.


    \item 设二维随机变量$(X,Y)$的概率密度为
    $$f(x,y)=\left\{\begin{array}{ll}
        e^{-y}, & 0<x<y,\\
        0, & \mbox{其他},
    \end{array}\right.$$
    求边缘概率密度.


    \item 设二维随机变量$(X,Y)$的概率密度为
    $$f(x,y)=\left\{\begin{array}{ll}
        cx^2y, & x^2\leq y \leq 1,\\
        0, & \mbox{其他},
    \end{array}\right.$$
    \begin{enumerate}
        \item 确定常数$c$;
        \item 求边缘概率密度.
    \end{enumerate}


    \item 将某一医药公司8月份和9月份收到的青霉素针剂的订货单数分别记为$X$和$Y$,据以往
    积累的资料知$X$和$Y$的联合分布律为

    \begin{table}[H]\centering
        \begin{tabular}{c|ccccc}
        \hline
        \diagbox{$Y$}{$X$}   & 51   & 52   & 53   & 54   & 55   \\ \hline
        51 & 0.06 & 0.05 & 0.05 & 0.01 & 0.01 \\
        52 & 0.07 & 0.05 & 0.01 & 0.01 & 0.01 \\
        53 & 0.05 & 0.10 & 0.10 & 0.05 & 0.05 \\
        54 & 0.05 & 0.02 & 0.01 & 0.01 & 0.03 \\
        55 & 0.05 & 0.06 & 0.05 & 0.01 & 0.03 \\ \hline
        \end{tabular}
    \end{table}
    \begin{enumerate}
        \item 求边缘分布律.
        \item 求8月份的订单数为51时,9月份订单数的条件分布律.
    \end{enumerate}
    
    
    \item 以$X$记某医院一天出生的婴儿的个数,$Y$记其中男婴的个数.设$X$和$Y$的联合分布律为
    \begin{equation}
        \begin{aligned}
        \nonumber
        P\{X=n,Y=m\}=&\frac{e^{-14}(7.14)^m(6.86)^{n-m}}{m!(n-m)!},\\
        &\qquad m=0,1,2,\cdots,n;\quad n=0,1,2,\cdots
        \end{aligned}
     \end{equation}
     \begin{enumerate}
         \item 求边缘分布律.
         \item 求条仵分布律.
         \item 特别地,写出当$X=20$时,$Y$的条件分布律.
     \end{enumerate}


     \item 设随机变量$X$在1,2,3,4四个整数中等可能地取一个值,另一个随机变量
     $Y$在$1\sim X$中等可能地取一整数值。求条件分布律$P\{Y=k|X=i\}$。


     \item 在第9题中
     \begin{enumerate}
         \item 求条件概率密度$f_{X|Y}(x|y)$.特别地,写出当$Y=\dfrac{1}{2}$时$X$的条件概率密度.
         \item 求条件概率密度$f_{Y|X}(y|x)$.特别地,写出当$X=\dfrac{1}{3},X=\dfrac{1}{2}$时$Y$的条件概率密度.
         \item 求条件概率
         $$P\left\{Y\geq \frac{1}{4}\left|X=\frac{1}{2}\right.\right\},\quad P\left\{Y\geq \frac{3}{4}\left|X=\frac{1}{2}\right.\right\}$$
     \end{enumerate}


     \item 设随机变量$(X,Y)$的概率密度为
     $$f(x,y)=\left\{\begin{array}{ll}
         1, & |y|<x,0<x<1\\
         0, & \mbox{其他}
     \end{array}\right.$$
     求条件概率密度$f_{Y|X}(y|x),f_{X|Y}(x|y)$.


     \item 设随机变量$X\sim U(0,1)$,当给定$X=x$时,随机变量$Y$的条件概率密度为
     $$f_{Y|X}(y|x)=\left\{\begin{array}{ll}
         x, & 0<y<\dfrac{1}{x} \\
         0, & \mbox{其他}
     \end{array}\right.$$
     \begin{enumerate}
         \item 求$X$和$Y$的联合概率密度$f(x,y)$.
         \item 求边缘密度$f_Y(y)$,并画出它的图形.
         \item 求$P\{X>Y\}$.
     \end{enumerate}


     \item \begin{enumerate}
         \item 问第1题中的随机变量$X$和$Y$是否相互独立?
         \item 间第14题中的随机变量$X$和$Y$是否相互独立(需说明理由)?
     \end{enumerate}


     \item \begin{enumerate}
         \item 设随机变量$(X,Y)$具有分布函数
         $$F(x,y)=\left\{\begin{array}{ll}
             (1-e^{-\alpha x})y, & x\geq 0,0\leq y \leq 1,\\
             1-e^{-\alpha x}, & x\geq 0,y>1,\\
             0, & \mbox{其他}
         \end{array}\right.\quad \alpha>0$$
         证明$X,Y$相互独立.
         \item 设随机变量$(X,Y)$具有分布律
         $$P\{X=x,Y=y\}=p^2(1-p)^{x+y-2},0<p<1,x,y\mbox{均为正整数}$$
         问$X,Y$是否相互独立。
     \end{enumerate}



     \item 设$X$和$Y$是两个相互独立的随机变量,$X$在区间$(0,1)$上服从均匀分布,$Y$的概率密度为
     $$f_Y(y)=\left\{\begin{array}{ll}
         \dfrac{1}{2} e^{-y/2}, & y>0\\
         0, & y\leq 0
     \end{array}\right.$$
     \begin{enumerate}
         \item 求$X$和$Y$的联合概率密度。
         \item 设有$a$的二元一次方程为$a^2+2Xa+Y=0$,试求$a$有实根的概率。
     \end{enumerate}



     \item 进行打靶,设弹着点$A(X,Y)$的坐标$X$和$Y$相互独立,且都服从$N(0,1)$分布,规定
    \begin{equation}
        \begin{aligned}
        \nonumber
        &\text{点}A\mbox{落在区域}D_1=\{(x,y)|x^2+y^2\leq 1\}\mbox{得}2\mbox{分}\\
        &\mbox{点}A\mbox{落在区域}D_2=\{(x,y)|1<x^2+y^2\leq 4\}\mbox{得}1\mbox{分}\\
        &\mbox{点}A\mbox{落在区域}D_3=\{(x,y)|x^2+y^2>4\}\mbox{得}0\mbox{分}\\              
        \end{aligned}
    \end{equation}
    以$Z$记打靶的得分.写出$X,Y$的联合概率密度,并求$Z$的分布律.


    \item 设$X$和$Y$是相互独立的随机变量,其概率密度分别为
    $$f_X(x)=\left\{\begin{array}{ll}
        \lambda e^{-\lambda x}, & x>0,\\
        0, & x\leq 0,
    \end{array}\right.\quad
    f_Y(y)=\left\{\begin{array}{ll}
        \mu e^{-\mu y}, & y<0,\\
        0, & y\leq 0
    \end{array}\right.$$
    其中$\lambda>0,\mu>0$是常数。引入随机变量
    $$Z=\left\{\begin{array}{ll}
        1, & \mbox{当}X\leq Y,\\
        0, & \mbox{当}X>Y
    \end{array}\right.$$
    \begin{enumerate}
        \item 求条件概率密度$f_{X|Y}(x|y)$.
        \item 求$Z$的分布律和分布函数.
    \end{enumerate}
     


    \item 设随机变量$(X,Y)$的概率密度为
    $$f(x,y)=\left\{\begin{array}{ll}
        x+y, & 0<x<1,0<y<1\\
        0, & \mbox{其他}
    \end{array}\right.$$
    分别求(1)$Z=X+Y$,(2)$Z=XY$的概率密度


    \item 设$X$和$Y$是两个相互独立的随机变量,其概率密度分别为
    $$f_X(x)=\left\{\begin{array}{ll}
        1, & 0\leq x\leq 1,\\
        0, & \mbox{其他},
    \end{array}\right.\quad
    f_Y(y)=\left\{\begin{array}{ll}
        e^{-y}, & y>0,\\
        0, & \mbox{其他},
    \end{array}\right.$$
    求随机变量$Z=X+Y$的概率密度。

     
    \item 某种商品一周的需求量是一个随机变量.其概率密度为
    $$f(t)=\left\{\begin{array}{ll}
        te^{-t}, & t>0\\
        0, & t\leq 0 
    \end{array}\right.$$
    设各周的需求量是相互独立的。求(1)两周,(2)三周的需求量的概率密度.


    \item 设随机变量$(X,Y)$的概率密度为
    $$f(x,y)=\left\{\begin{array}{ll}
        \dfrac{1}{2}(x+y)e^{-(x+y)}, & x>0,y>0\\
        0, & \mbox{其他}
    \end{array}\right.$$
    \begin{enumerate}
        \item 问$X$和$Y$是否相互独立
        \item 求$Z=X+Y$的概率密度
    \end{enumerate}


    \item 设随机变量$X,Y$相互独立,且具有相同的分布,它们的概率密度均为
    $$f(x)=\left\{\begin{array}{ll}
        e^{1-x}, & x>1\\
        0, & \mbox{其他}
    \end{array}\right.$$
    求$Z=X+Y$的概率密度


    \item 设随机变量$X,Y$相互独立,它们的概率密度均为
    $$f(x)=\left\{\begin{array}{ll}
        e^{-x}, & x>0\\
        0, & \mbox{其他}
    \end{array}\right.$$
    求$Z=Y/X$的概率密度


    \item 设随机变量$X,Y$相互独立,它们都在区间(0,1)上服从均匀分布,$A$是以$X,Y$为边
    长的矩形的面积,求$A$的概率密度.


    \item 设$X,Y$是相互独立的随机变量.它们都服从正态分布$N(0,\sigma^2)$.试验证随机变量$Z=\sqrt{X^2+Y^2}$
    的概率密度为
    $$f_Z(z)=\left\{\begin{array}{ll}
        \dfrac{z}{\sigma^2}e^{-z^2/(2\sigma^2)}, & z\geq 0\\
        0, & \mbox{其他}
    \end{array}\right.$$
    我们称$Z$服从参数为$\sigma(\sigma>0)$的瑞利分布.



    \item 设随机变量$(X,Y)$的概率密度为
    $$f(x,y)=\left\{\begin{array}{ll}
        be^{-(x+y)}, & 0<x<1,0<y<\infty\\
        0, & \mbox{其他}
    \end{array}\right.$$
    \begin{enumerate}
        \item 试确定常数$b$.
        \item 求边缘概率密度$f_X(x),f_Y(y)$.
        \item 求函数$U=\max\{X,Y\}$.
    \end{enumerate}
   


    \item 设某种型号的电子元件的寿命(以小时计)近似地服从正态分布$N(160,20^2)$.随机
    地选取4只,求其中没有一只寿命小于180的概率.



    \item 对某种电子装置的输出测量了5次.得到结果为$X_1,X_2,X_3,X_4,X_5$.设它们是相互
    独立的随机变量且都服从参数$\sigma=2$的瑞利分布.
    \begin{enumerate}
        \item 求$Z=\max \{X_1,X_2,X_3,X_4,X_5\}$的分布函数
        \item 求$P\{Z>4\}$
    \end{enumerate}


    \item 设随机变量$X,Y$相互独立,且服从同一分布,试证明;
    $$P\{a<\min \{X,Y\} \leq b\}={[P\{X>a\}]}^2-{[P\{X>b\}]}^2\quad (a\leq b)$$


    \item 设$X,Y$是相互独立的随机变量.其分布律分别为
    $$P\{X=k\}=p(k),\quad k=0,1,2,\cdots,$$
    $$P\{Y=r\}=q(r),\quad r=0,1,2,\cdots.$$
    证明随机变量$Z=X+Y$的分布律为
    $$P\{Z=i\}=\sum _{k=0}^i p(k)q(i-k),\quad i=0,1,2,\cdots$$


    \item 设$X,Y$是相互独立的随机变量,$X\sim \pi(\lambda_1),Y\sim \pi(\lambda_2)$.证明$Z=X+Y\sim \pi(\lambda_1+\lambda_2)$
    
    
    
    \item 设$X,Y$是相互独立的随机变量,$X\sim b(n_1,p),Y\sim b(n_2,p)$.证明$Z=X+Y\sim b(n_1+n_2,p)$
    

    \item 设随机变量$(X,Y)$的分布律为  
    \begin{table}[H]\centering
        \begin{tabular}{c|cccccc}
        \hline
        \diagbox{$Y$}{$X$}  & 0    & 1    & 2    & 3    & 4    & 5    \\ \hline
        0 & 0.00 & 0.01 & 0.03 & 0.05 & 0.07 & 0.09 \\
        1 & 0.01 & 0.02 & 0.04 & 0.05 & 0.06 & 0.08 \\
        2 & 0.01 & 0.03 & 0.05 & 0.05 & 0.05 & 0.06 \\
        3 & 0.01 & 0.02 & 0.04 & 0.06 & 0.06 & 0.05 \\ \hline
        \end{tabular}
    \end{table}
    \begin{enumerate}
        \item 求$P\{X=2|Y=2\},P\{Y=3|X=0\}$
        \item 求$V=\max \{X,Y\}$的分布律
        \item 求$U=\min \{X,Y\}$的分布律
        \item 求$W=X+Y$的分布律
    \end{enumerate}
  

\end{enumerate}
\end{document}