\documentclass[10pt,a4paper]{article}
\usepackage[UTF8]{ctex}
\usepackage{fontspec}
\usepackage{geometry} 
\usepackage{amsmath}
\usepackage[shortlabels]{enumitem}
\usepackage{float}
\usepackage{graphicx}
\usepackage{subfigure}
\usepackage{epstopdf}
\usepackage{amsmath,amssymb}
\usepackage{diagbox}
\DeclareSymbolFont{EulerExtension}{U}{euex}{m}{n}
\DeclareMathSymbol{\euintop}{\mathop} {EulerExtension}{"52}
\DeclareMathSymbol{\euointop}{\mathop} {EulerExtension}{"48}
\let\intop\euintop
\let\ointop\euointop

\geometry{left=3.17cm,right=3.17cm,top=2.53cm,bottom=2.54cm}
%\setmainfont{Times New Roman}
\pagestyle{plain}
\setlist[enumerate,1]{label=\textbf{\arabic*.}}
\setlist[enumerate,2]{label=(\arabic*)}

\begin{document}

\begin{enumerate}




    \item 在一箱子中装有12只开关.其中2只是次品,在其中取两次.每次任取一只,考虑两
    种试验:(1)放回抽样;(2)不放回抽样.我们定义随机变量$X,Y$如下:
    $$X=\left\{\begin{array}{ll}
        0, & \mbox{若第一次取出的是正品}\\
        1, & \mbox{若第一次取出的是次品}
    \end{array}\right.\\
    Y=\left\{\begin{array}{ll}
        0, & \mbox{若第二次取出的是正品}\\
        1, & \mbox{若第二次取出的是次品}\\
    \end{array}\right.$$
    试分别就(1)、(2)两种情况.写出$X$和$Y$的联合分布律.


    \item \begin{enumerate}
        \item 盒子里装有3只黑球、2只红球、2只白球,在其中任取4只球.以$X$表示取到黑
        球的只数,以$Y$表示取到红球的只数.求$X$和$Y$的联合分布律.
        \item 在(1)中求$P\{X>Y\},P\{Y=2X\},P\{X+Y=3\},P\{X<3-Y\}$.
    \end{enumerate}


    \item 设随机变量$X,Y$的概率密度为
    $$f(x,y)=\left\{\begin{array}{ll}
        k(6-x-y), & 0<x<2,2<y<4\\
        0, & \mbox{其他}
    \end{array}\right.$$
    \begin{enumerate}
        \item 确定常数$k$.
        \item 求$P\{X<1,Y<3\}$.
        \item 求$P\{X<1.5\}$.
        \item 求$P\{X+Y\leq 4\}$.
    \end{enumerate}


    \item 设$X,Y$都是非负的连续型随机变量,它们相互独立.
    \begin{enumerate}
        \item 证明$$P\{X<Y\}=\int _0^\infty F_X(x)f_Y(x)\mathrm{d}x$$
        其中$F_X(x)$是$X$的分布函数,$f_Y(y)$是$Y$的概率密度.
        \item 设$X,Y$相互独立.其概率密度分别为
        $$f_X(x)=\left\{\begin{array}{ll}
            \lambda_1 e^{-\lambda_1 x}, & x>0,\\
            0, & \mbox{其他},
        \end{array}\right.\\
        f_Y(y)=\left\{\begin{array}{ll}
            \lambda_2 e^{-\lambda_2 y}, & y>0,\\
            0, & \mbox{其他},
        \end{array}\right.$$
        求$P\{X<Y\}$.
    \end{enumerate}


    \item 设随机变量$(X,Y)$具有分布函数
    $$F(x,y)=\left\{\begin{array}{ll}
        1-e^{-x}-e^{-y}+e^{-x-y}, & x>0,y>0\\
        0, & \mbox{其他}
    \end{array}\right.$$
    求边缘分布函数.


    \item 将一枚硬币掷3次以$X$表示前2次中出现$H$的次数,以$Y$表示3次中出现$H$的次
    数.求$X,Y$的联合分布律以及$(X,Y)$的边缘分布律.


    \item 设二维随机变量$(X,Y)$的概率密度为
    $$f(x,y)=\left\{\begin{array}{ll}
        4.8y(2-x), & 0\leq x \leq 1,0\leq y \leq x,\\
        0, & \mbox{其他},
    \end{array}\right.$$
    求边缘概率密度.


    \item 设二维随机变量$(X,Y)$的概率密度为
    $$f(x,y)=\left\{\begin{array}{ll}
        e^{-y}, & 0<x<y,\\
        0, & \mbox{其他},
    \end{array}\right.$$
    求边缘概率密度.


    \item 设二维随机变量$(X,Y)$的概率密度为
    $$f(x,y)=\left\{\begin{array}{ll}
        cx^2y, & x^2\leq y \leq 1,\\
        0, & \mbox{其他},
    \end{array}\right.$$
    \begin{enumerate}
        \item 确定常数$c$;
        \item 求边缘概率密度.
    \end{enumerate}


    \item 将某一医药公司8月份和9月份收到的青霉素针剂的订货单数分别记为$X$和$Y$,据以往
    积累的资料知$X$和$Y$的联合分布律为

    \begin{table}[H]\centering
        \begin{tabular}{c|ccccc}
        \hline
        \diagbox{$Y$}{$X$}   & 51   & 52   & 53   & 54   & 55   \\ \hline
        51 & 0.06 & 0.05 & 0.05 & 0.01 & 0.01 \\
        52 & 0.07 & 0.05 & 0.01 & 0.01 & 0.01 \\
        53 & 0.05 & 0.10 & 0.10 & 0.05 & 0.05 \\
        54 & 0.05 & 0.02 & 0.01 & 0.01 & 0.03 \\
        55 & 0.05 & 0.06 & 0.05 & 0.01 & 0.03 \\ \hline
        \end{tabular}
    \end{table}
    \begin{enumerate}
        \item 求边缘分布律.
        \item 求8月份的订单数为51时,9月份订单数的条件分布律.
    \end{enumerate}
    
    
    \item 以$X$记某医院一天出生的婴儿的个数,$Y$记其中男婴的个数.设$X$和$Y$的联合分布律为
    \begin{equation}
        \begin{aligned}
        \nonumber
        P\{X=n,Y=m\}=&\frac{e^{-14}(7.14)^m(6.86)^{n-m}}{m!(n-m)!},\\
        &\qquad m=0,1,2,\cdots,n;\quad n=0,1,2,\cdots
        \end{aligned}
     \end{equation}
     \begin{enumerate}
         \item 求边缘分布律.
         \item 求条仵分布律.
         \item 特别,写出当$X=20$时,$Y$的条件分布律.
     \end{enumerate}


     \item 设随机变量$X$在1,2,3,4四个整数中等可能地取一个值,另一个随机变量
     $Y$在$1\sim X$中等可能地取一整数值。求条件分布律$P\{Y=k|X=i\}$。





 

   





    
    % \begin{table}[H]\centering
    %     \begin{tabular}{c|cccccc}
    %     \hline
    %     \diagbox{$Y$}{$X$}  & 0    & 1    & 2    & 3    & 4    & 5    \\ \hline
    %     0 & 0.00 & 0.01 & 0.03 & 0.05 & 0.07 & 0.09 \\
    %     1 & 0.01 & 0.02 & 0.04 & 0.05 & 0.06 & 0.08 \\
    %     2 & 0.01 & 0.03 & 0.05 & 0.05 & 0.05 & 0.06 \\
    %     3 & 0.01 & 0.02 & 0.04 & 0.06 & 0.06 & 0.05 \\ \hline
    %     \end{tabular}
    % \end{table}

  

\end{enumerate}
\end{document}