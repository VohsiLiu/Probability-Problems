\documentclass[10pt,a4paper]{article}
\usepackage[UTF8]{ctex}
\usepackage{fontspec}
\usepackage{geometry} 
\usepackage{amsmath}
\usepackage[shortlabels]{enumitem}
\usepackage{float}
\usepackage{graphicx}
\usepackage{subfigure}
\usepackage{epstopdf}
\usepackage{amsmath,amssymb}
\usepackage{diagbox}
\usepackage{setspace}
\usepackage{enumitem}
\usepackage[table,xcdraw]{xcolor}
\DeclareSymbolFont{EulerExtension}{U}{euex}{m}{n}
\DeclareMathSymbol{\euintop}{\mathop} {EulerExtension}{"52}
\DeclareMathSymbol{\euointop}{\mathop} {EulerExtension}{"48}
\let\intop\euintop
\let\ointop\euointop

\geometry{left=3.17cm,right=3.17cm,top=2.53cm,bottom=2.54cm}
%\setmainfont{Times New Roman}
\pagestyle{plain}
\setlist[enumerate,1]{label=\textbf{\arabic*.}}
\setlist[enumerate,2]{label=(\arabic*)}

\begin{document}

\begin{enumerate}


    \item 某批矿砂的5个样品中的镍含量,经测定为(\%)
    $$\begin{array}{ccccc}
        3.25 & 3.27 &  3.24 & 3.26 & 3.24
    \end{array}$$
    设测定值总体服从正态分布,但参数均未知,问在$\alpha=0.01$下能否接受假设:这批矿砂的镍含
    量的均值为3.25.


    \begin{spacing}{1.5}
    \item 如果一个矩形的宽度$w$与长度$l$的比$\dfrac{w}{l}=\dfrac{1}{2}(\sqrt{5}-1)\approx 0.618$,这样的矩形称为黄金矩形.
    \end{spacing}
    这种尺寸的矩形使人们看上去有良好的感觉.现代的建筑构件(如窗架)、工艺品(如
    图片镜框),甚至司机的执照、商业的信用卡等常常都是采用黄金矩形.下面列出某工艺品工
    厂随机取的20个矩形的宽度与长度的比值:
    $$\begin{array}{cccccccccc}
        0.693 & 0.749 & 0.654 & 0.670 & 0.662  & 0.672 &  0.615 &  0.606 & 0.690 & 0.628\\
        0.668 & 0.611 & 0.606 & 0.609  & 0.601 & 0.553 & 0.570  & 0.844 &  0.576 &  0.933 
    \end{array}$$
    设这一工厂生产的矩形的宽度与长度的比值总体服从正态分布,其均值为$\mu$,方差为$\sigma^2$,$\mu,\sigma^2$
    均未知.试检验假设(取$\alpha=0.05$)
    $$H_0:\mu=0.618,\quad H_1:\mu\neq 0.618$$






    \item 要求一种元件平均使用寿命不得低于1000$\, \mathrm{h}$,生产者从一批这种元件中随机抽取25
    件,测得其寿命的平均值为950$\, \mathrm{h}$.已知该种元件寿命服从标准差为$\sigma=100\, \mathrm{h}$的正态分布.试
    在显著性水平$\alpha=0.05$下判断这批元件是否合格?设总体均值为$\mu$,$\mu$未知.即需检验假设
    $H_0:\mu\geq 1000,H_1:\mu<1000$.




    \item 下面列出的是某工厂随机选取的20只部件的装配时间(min):
    $$\begin{array}{cccccccccc}
        9.8 &  10.4 &  10.6 &  9.6 &  9.7 &  9.9  & 10.9 &  11.1 &  9.6 &  10.2\\
        10.3 &  9.6 &  9.9 &  11.2 &  10.6 &  9.8  &  10.5 &  10.1 &  10.5 &  9.7
    \end{array}$$
    设装配时间的总体服从正态分布$N(\mu,\sigma^2)$,$\mu,\sigma^2$均未知.是否可以认为装配时间的均值显著
    大于10(取$\alpha=0.05$)?





    \item 按规定,100$\, $g罐头番茄汁中的平均维生素C含量不得少于21$\, $mg/g.现从工厂的产
    品中抽取17个罐头,其100$\, $g番茄汁中,测得维生素C含量(mg/g)记录如下:
    $$\begin{array}{ccccccccccccccccc}
        16 & 25 &  21 & 20 & 23 & 21 & 19 & 15 & 13 & 23 & 17 & 20 & 29 & 18 & 22 & 16 &  22
    \end{array}$$
    设维生素含量服从正态分布$N(\mu,\sigma^2)$,$\mu,\sigma^2$均未知,问这批罐头是否符合要求(取显著性水平
    $\alpha=0.05$).





    \item 下表分别给出两位文学家马克$\cdot$吐温的8篇小品文以及斯诺特格拉斯
    的10篇小品文中由3个字母组成的单字的比例.
    \renewcommand{\arraystretch}{1.3}
    \begin{table}[H]\centering
        \begin{tabular}{c|cccccccccc}
        马克$\cdot$吐温 & 0.225 & 0.262 & 0.217 & 0.240 & 0.230 & 0.229 & 0.235 & 0.217 &  &  \\ \hline
        斯诺特格拉斯 & 0.209 & 0.205 & 0.196 & 0.210 & 0.202 & 0.207 & 0.224 & 0.223 & 0.220 & 0.201
    \end{tabular}
    \end{table}
    \renewcommand{\arraystretch}{1.0}
    设两组数据分别来自正态总体,且两总体方差相等,但参数均未知,两样本相互独立.问两位
    作家所写的小品文中包含由3个字母组成的单字的比例是否有显著的差异(取$\alpha=0.05$)?




    \item 在20世纪70年代后期人们发现,在酿造啤酒时,在麦芽干燥过程中形成致癌物质亚
    硝基二甲胺(NDMA)到了20世纪80年代初期开发了一种新的麦芽干燥过程下面给出分
    别在新老两种过程中形成的NDMA含量(以10亿份中的份数计):
    \renewcommand{\arraystretch}{1.3}
    \begin{table}[H]\centering
    \begin{tabular}{c|cccccccccccc}
    老过程 & 6 & 4 & 5 & 5 & 6 & 5 & 5 & 6 & 4 & 6 & 7 & 4 \\ \hline
    新过程 & 2 & 1 & 2 & 2 & 1 & 0 & 3 & 2 & 1 & 0 & 1 & 3
    \end{tabular}
    \end{table}
    \renewcommand{\arraystretch}{1.0}
    设两样本分别来自正态总体,且两总体的方差相等,但参数均未知.两样本独立.分别以$\mu_1,\mu_2$
    记对应于老、新过程的总体的均值,试检验假设($\alpha=0.05$)
    $$H_0:\mu_1-\mu_2\leq 2,\quad H_1:\mu_1-\mu_2>2$$




    \item 随机地选了8个人,分别测量了他们在早晨起床时和晚上就寝时的身高(cm)、得到以
    下的数据
    \renewcommand{\arraystretch}{1.3}
    \begin{table}[H]\centering
        \begin{tabular}{c|cccccccc}
        \hline
        序号                             & 1                       & 2                       & 3                       & 4   & 5   & 6   & 7   & 8   \\ \hline
        早上($x_i$)                      & 172                     & 168                     & 180                     & 181 & 160 & 163 & 165 & 177 \\ \hline
        \multicolumn{1}{l|}{晚上($y_i$)} & \multicolumn{1}{l}{172} & \multicolumn{1}{l}{167} & \multicolumn{1}{l}{177} & 179 & 159 & 161 & 166 & 175 \\ \hline
        \end{tabular}
    \end{table}
    \renewcommand{\arraystretch}{1.0}
    设各对数据的差$D_i=X_i-Y_i(i=1,2,\cdots,8)$是来自正态总体$N(\mu_D,\sigma^2_D)$的样本,$\mu_D,\sigma^2_D$均未
    知.问是否可以认为早晨的身高比晚上的身高要高(取$\alpha=0.05$)?




    \item 为了比较用来做鞋子后跟的两种材料的质量,选取了15名男子(他们的生活条件各
    不相同),每人穿一双新鞋,其中一只是以材料A做后跟,另一只以材料B做后跟,其厚度均
    为10$\, $mm.过了一个月再测量厚度,得到数据如下:
    \renewcommand{\arraystretch}{1.3}
    \begin{table}[H]\centering
        \begin{tabular}{c|ccccccccccccccc}
        \hline
        男子                              & 1                       & 2                       & 3                       & 4   & 5   & 6   & 7   & 8   & 9   & 10  & 11  & 12  & 13  & 14  & 15  \\ \hline
        材料$A(x_i)$                      & 6.6                     & 7.0                     & 8.3                     & 8.2 & 5.2 & 9.3 & 7.9 & 8.5 & 7.8 & 7.5 & 6.1 & 8.9 & 6.1 & 9.4 & 9.1 \\ \hline
        \multicolumn{1}{l|}{材料$B(y_i)$} & \multicolumn{1}{l}{7.4} & \multicolumn{1}{l}{5.4} & \multicolumn{1}{l}{8.8} & 8.0 & 6.8 & 9.1 & 6.3 & 7.5 & 7.0 & 6.5 & 4.4 & 7.7 & 4.2 & 9.4 & 9.1 \\ \hline
        \end{tabular}
    \end{table}
    \renewcommand{\arraystretch}{1.0}
    设$D_i=X_i-Y_i(i=1,2,\cdots,15)$是来自正态总体$N(\mu_D,\sigma^2_D)$的样本,$\mu_D,\sigma^2_D$均未知.问是否可
    以认为以材料A制成的后跟比材料B的耐穿(取$\alpha=0.05$)?




    \item 为了试验两种不同的某谷物的种子的优劣,选取了10块土质不同的土地,并将每块
    土地分为面积相同的两部分,分别种植这两种种子.设在每块土地的两部分人工管理等条件
    完全一样.下面给出各块土地上的单位面积产量:
    \renewcommand{\arraystretch}{1.3}
    \begin{table}[H]\centering
        \begin{tabular}{c|cccccccccc}
        \hline
        土地编号$i$    & 1  & 2  & 3  & 4  & 5  & 6  & 7  & 8  & 9  & 10 \\ \hline
        种子$A(x_i)$ & 23 & 35 & 29 & 42 & 39 & 29 & 37 & 34 & 35 & 28 \\ \hline
        种子$B(y_i)$ & 26 & 39 & 35 & 40 & 38 & 24 & 36 & 27 & 41 & 27 \\ \hline
        \end{tabular}
    \end{table}
    \renewcommand{\arraystretch}{1.0}
    设$D_i=X_i-Y_i(i=1,2,\cdots,10)$是来自正态总体$N(\mu_D,\sigma^2_D)$的样本,$\mu_D,\sigma^2_D$均未知.问以这两
    种种子种植的谷物的产量是否有显著的差异(取$\alpha=0.05$)?




    \item 一种混杂的小麦品种,株高的标准差为$\sigma_0=14\, \mathrm{cm}$.经提纯后随机抽取10株,它们的
    株高(以cm计)为
    $$\begin{array}{cccccccccc}
        90 & 105 & 101 & 95  & 100 & 100 & 101 & 105 &  93 & 97
    \end{array}$$
    考察提纯后群体是否比原群体整齐?取显著性水平$\alpha=0.01$,并设小麦株高服从$N(\mu,\sigma^2)$.




    \item 某种导线,要求其电阻的标准差不得超过$0.005\, \Omega$ ,今在生产的一批导线中取样品9
    根,测得$s=0.007\, \Omega$,设总体为正态分布,参数均未知.问在显著性水平$\alpha=0.05$下能否认为
    这批导线的标准差显著地偏大?



    \item 在第2题中记总体的标准差为$\sigma$ ,试检验假设(取$\alpha=0.05$)
    $$H_0:\sigma^2=0.11^2,\quad H_1:\sigma^2\neq 0.11^2$$




    \item 测定某种溶液中的水分,它的10个测定值给出$s=0.037\%$ ,设测定值总体为正态分
    布,$\sigma^2$为总体方差,$\sigma^2$未知.试在显著性水平$\alpha=0.05$下检验假设
    $$H_0:\sigma\geq 0.04\%,\quad H_1:\sigma<0.04\%$$




    \item 在第6题中分别记两个总体的方差为$\sigma_1^2,\sigma_2^2$.试检验假设(取$\alpha=0.05$)
    $$H_0:\sigma_1^2=\sigma_2^2, \quad H_1:\sigma_1^2\neq \sigma^2_2$$
    以说明在第6题中我们假设$\sigma_1^2=\sigma_2^2$是合理的.





    \item 在第7题中分别记两个总体的方差为$\sigma_1^2,\sigma_2^2$.试检验假设(取$\alpha=0.05$)
    $$H_0:\sigma_1^2=\sigma_2^2, \quad H_1:\sigma_1^2\neq \sigma^2_2$$
    以说明在第7题中我们假设$\sigma_1^2=\sigma_2^2$是合理的.




    \item 两种小麦品种从播种到抽穗所需的天数如下:
    \renewcommand{\arraystretch}{1.3}   
    \begin{table}[H]\centering
        \begin{tabular}{c|cccccccccc}
        $x$ & 101 & 100 & 99  & 99 & 98 & 100 & 98 & 99 & 99 & 99  \\ \hline
        $y$ & 100 & 98  & 100 & 99 & 98 & 99  & 98 & 98 & 99 & 100
        \end{tabular}
    \end{table}
    \renewcommand{\arraystretch}{1.0}
    设两样本依次来自正态总体$N(\mu_1,\sigma_1^2),N(\mu_2,\sigma_2^2)$,$\mu_i,\sigma_i(i=1,2)$均未知,
    两样本相互独立.
    \begin{enumerate}
        \item 试检验假设$H_0:\sigma_1^2=\sigma_2^2,H_1:\sigma_1^2\neq \sigma_2^2$(取$\alpha=0,05$)
        \item 若能接受$H_0$,接着检验假设$H^\prime_0:\mu_1=\mu_2,H^\prime_1:\mu_1\neq \mu_2$ (取$\alpha=0.05$).
    \end{enumerate}




    \item 用一种叫“混乱指标”的尺度去衡量工程师的英语文章的可理解性,对混乱指标的打
    分越低表示可理解性越高.分别随机选取13篇刊载在工程杂志上的论文,以及10篇未出版
    的学术报告,对它们的打分列于下表:
    \renewcommand{\arraystretch}{1.3} 
    \begin{table}[H]\centering
    \begin{tabular}{cccc|cccc}
        \hline
        \multicolumn{4}{c|}{工程杂志上的论文(数据\uppercase\expandafter{\romannumeral1})} & \multicolumn{4}{c}{未出版的学术报告(数据\uppercase\expandafter{\romannumeral2})}         \\ \hline
        1.79   & 1.75   & 1.67   & 1.65   & 2.39      & 2.51      & 2.86     & {\color[HTML]{FFFFFF} 2.86} \\
        1.87   & 1.74   & 1.94   &        & 2.56      & 2.29      & 2.49     & {\color[HTML]{FFFFFF} 2.86} \\
        1.62   & 2.06   & 1.33   &        & 2.36      & 2.58      &          & {\color[HTML]{FFFFFF} 2.49} \\
        1.96   & 1.69   & 1.70   &        & 2.62      & 2.41      &          & {\color[HTML]{FFFFFF} 2.49} \\ \hline
    \end{tabular}
    \end{table}
    \renewcommand{\arraystretch}{1.0} 
    设数据\uppercase\expandafter{\romannumeral1},\uppercase\expandafter{\romannumeral2}分别来自正态总体$N(\mu_1,\sigma_1^2),N(\mu_2,\sigma_2^2)$
    ,$\mu_1,\mu_2,\sigma_1^2,\sigma_2^2$均未知,两样本独立.
    \begin{enumerate}
        \item 试检验假设$H_0:\sigma_1^2=\sigma_2^2,H_1:\sigma_1^2\neq \sigma_2^2$(取$\alpha=0.1$).
        \item 若能接受$H_0$,接着检验假设$H^\prime_0:\mu_1=\mu_2,H^\prime_1:\mu_1\neq \mu_2$ (取$\alpha=0.1$) .
    \end{enumerate}




    \item 有两台机器生产金属部件.分别在两台机器所生产的部件中各取一容量$n_1=60,n_2=40$
    的样本,测得部件重量(以kg计)的样本方差分别为$s_1^2=15.46,s_2^2=9.66$.设两样本相互
    独立.两总体分别服从$N(\mu_1,\sigma_1^2),N(\mu_2,\sigma_2^2)$分布,$\mu_i,\sigma_i^2(i=1,2)$均未知.试在显著性水平
    $\alpha=0.05$下检验假设
    $$H_0:\sigma_1^2\leq \sigma_2^2,\quad H_1:\sigma_1^2>\sigma_2^2$$




    \item 设需要对某一正态总体的均值进行假设检验
    $$H_0:\mu \geq 15,\quad H_1:\mu <15$$
    已知$\sigma^2=2.5$.取$\alpha=0.05$.若要求当$H_1$中的$\mu\leq 13$时犯第\uppercase\expandafter{\romannumeral2}类错误的概率不超过$\beta=0.05$,
    求所需的样本容量.




    \item 电池在货架上滞留的时间不能太长.下面给出某商店随机选取的8只电池的货架滞
    留时间(以天计):
    $$\begin{array}{cccccccc}
        108 & 124 & 124 & 106 & 138 & 163 & 159 & 134
    \end{array}$$
    设数据来自正态总体$N(\mu,\sigma^2),\mu,\sigma^2$未知.
    \begin{enumerate}
        \item 试检验假设$H_0:\mu \leq 125,H_1:\mu >125$,取$\alpha=0.05$
        \item 若要求在上述$H_1$中$(\mu/125)\sigma\geq 1.4$时,犯第\uppercase\expandafter{\romannumeral2}类错误的概率不超过$\beta=0.1$,
        求所需的样本容量.
    \end{enumerate}




    \item 一药厂生产一种新的止痛片,厂方希望验证服用新药片后至开始起作用的时间间隔
    较原有止痛片至少缩短一半,因此厂方提出需检验假设
    $$H_0:\mu_1\leq 2\mu_2,\quad H_1:\mu_1>2\mu_2$$
    此处$\mu_1,\mu_2$分别是服用原有止痛片和服用新止痛片后至起作用的时间间隔的总体的均值.设
    两总体均为正态且方差分别为已知值$\sigma_1^2,\sigma_2^2$。现分别在两总体中取一样本$X_1,X_2,\cdots,X_{n_1}$和
    $Y_1,Y_2,\cdots,Y_{n_2}$,设两个样本独立.试给出上述假设$H_0$的拒绝域,取显著性水平为$\alpha$.




    \item 检查了一本书的100页,记录各页中印刷错误的个数,其结果为
    \renewcommand{\arraystretch}{1.3} 
    \begin{table}[H]\centering
        \begin{tabular}{c|cccccccc}
        错误个数$f_i$    & 0  & 1  & 2  & 3 & 4 & 5 & 6 & $\geq$7 \\ \hline
        含$f_i$个错误的页数 & 36 & 40 & 19 & 2 & 0 & 2 & 1 & 0      
        \end{tabular}
    \end{table}
    \renewcommand{\arraystretch}{1.0} 
    问能否认为一页的印刷错误的个数服从泊松分布(取$\alpha=0.05$).





    \item 在一批灯泡中抽取300只作寿命试验,其结果如下:
    \renewcommand{\arraystretch}{1.3}
    \begin{table}[H]\centering
        \begin{tabular}{c|cccc}
        寿命$t\ (\mathrm{h})$ & $0\leq t\leq 100$ & $100<t\leq 200$ & $200<t\leq 300$ & $t>300$ \\ \hline
        灯泡数              & 121               & 78              & 43              & 58     
        \end{tabular}
    \end{table}
    \renewcommand{\arraystretch}{1.0}
    取$\alpha=0.05$,试检验假设
    $H_0:$灯泡寿命服从指数分布
    $$f(t)=\left\{\begin{array}{ll}
        0.005e^{-0.005t}, & t\geq 0 \\
        0, & t<0
    \end{array}\right.$$



    \item 下面给出了随机选取的某大学一年级学生(200名)一次数学考试的成绩.
    \begin{enumerate}
        \item 画出数据的直方图.
        \item 试取$\alpha=0.1$检验数据来自正态总体$N(60,15^2)$.
    \end{enumerate}
    \renewcommand{\arraystretch}{1.3}
    \begin{table}[H]\centering
        \begin{tabular}{c|cccc}
        \hline
        分数$x$ & $20\leq x\leq 30$ & $30<x\leq 40$ & $40<x\leq 50$ & $50<x\leq 60$  \\ \hline
        学生数   & 5                 & 15            & 30            & 51             \\ \hline
        分数$x$ & $60<x\leq 70$     & $70<x\leq 80$ & $80<x\leq 90$ & $90<x\leq 100$ \\ \hline
        学生数   & 60                & 23            & 10            & 6              \\ \hline
        \end{tabular}
    \end{table}
    \renewcommand{\arraystretch}{1.0}





    \item 袋中装有8只球,其中红球数未知.在其中任取3只,记录红球的只数$X$,然后放回,
    再任取3只,记录红球的只数,然后放回.如此重复进行了112次,其结果如下:
    \renewcommand{\arraystretch}{1.3}
    \begin{table}[H]\centering
        \begin{tabular}{c|cccc}
        $X$ & 0 & 1  & 2  & 3  \\ \hline
        次数  & 1 & 31 & 55 & 25
        \end{tabular}
        \end{table}
    \renewcommand{\arraystretch}{1.0}
    试取$\alpha=0.05$检验假设
    $H_0:X$服从超几何分布
    $$P\{X=k\}=\dfrac{\binom{5}{k} \binom{3}{3-k}} {\binom{8}{3}} ,\quad k=0,1,2,3.$$
    即检验假设$H_0:$红球的只数为5.






    \item 一农场10年前在一鱼塘中按比例$20:15:40:25$投放了四种鱼:鲑鱼、鲈鱼、竹夹
    鱼和鲇鱼的鱼苗,现在在鱼塘里获得一样本如下:
    \renewcommand{\arraystretch}{1.3}
    \begin{table}[H]\centering
        \begin{tabular}{c|ccccc}
        \hline
        序号    & 1   & 2   & 3   & 4   &            \\ \hline
        种类    & 鲑鱼  & 鲈鱼  & 竹夹鱼 & 鲇鱼  &            \\ \hline
        数量(条) & 132 & 100 & 200 & 168 & $\sum=600$ \\ \hline
        \end{tabular}
    \end{table}
    \renewcommand{\arraystretch}{1.0}
    试取$\alpha=0.05$,检验各类鱼数量的比例较10年前是否有显著的改变.




    \item 某种鸟在起飞前,双足齐跳的次数$X$服从几何分布,其分布律为
    $$P\{X=x\}=p^{x-1}(1-p),\quad x=1,2,\cdots.$$
    今获得一样本如下:
    \renewcommand{\arraystretch}{1.3}
    \begin{table}[H]\centering
        \begin{tabular}{c|ccccccccccccc}
        $x$       & 1  & 2  & 3  & 4 & 5 & 6 & 7 & 8 & 9 & 10 & 11 & 12 & $\geq 13$ \\ \hline
        观察到$x$的次数 & 48 & 31 & 20 & 9 & 6 & 5 & 4 & 2 & 1 & 1  & 2  & 1  & 0        
        \end{tabular}
    \end{table}
    \renewcommand{\arraystretch}{1.0}
    \begin{enumerate}
        \item 求$p$的最大似然估计值.
        \item 取$\alpha=0.05$,检验假设:$H_0:$数据来自总体$P\{X=x\}=p^{x-1}(1-p),x=1,2,\cdots$
    \end{enumerate}





    \item 分别抽查了两球队部分队员行李的重量(kg)为:   
    \renewcommand{\arraystretch}{1.3}
    \begin{table}[H]\centering
        \begin{tabular}{c|cccccc}
        1队 & 34 & 39 & 41 & 28 & 33 &    \\ \hline
        2队 & 36 & 40 & 35 & 31 & 39 & 36
        \end{tabular}
    \end{table}
    \renewcommand{\arraystretch}{1.0}
    设两样本独立且1,2两队队员行李重量总体的概率密度至多差一个平移,记两总体的均值分
    别为$\mu_1,\mu_2$,且$\mu_1,\mu_2$均未知.试检验假设:$H_0:\mu_1=\mu_2,H_1:\mu_1<\mu_2$(取$\alpha=0.05$).




    



    \item 下面给出两种型号的计算器充电以后所能使用的时间(h):
    \renewcommand{\arraystretch}{1.3}
    \begin{table}[H]\centering
        \begin{tabular}{c|cccccccccccc}
        型号$A$ & 5.5 & 5.6 & 6.3 & 4.6 & 5.3 & 5.0 & 6.2 & 5.8 & 5.1 & 5.2 & 5.9 &     \\ \hline
        型号$B$ & 3.8 & 4.3 & 4.2 & 4.0 & 4.9 & 4.5 & 5.2 & 4.8 & 4.5 & 3.9 & 3.7 & 4.6
        \end{tabular}
    \end{table}
    \renewcommand{\arraystretch}{1.0}
    设两样本独立且数据所属的两总体的概率密度至多差一个平移,试问能否认为型号$A$的计
    算器平均使用时间比型号$B$来得长($\alpha=0.05$)





    \item 下面给出两个工人五天生产同一种产品每天生产的件数:
    \renewcommand{\arraystretch}{1.3}
    \begin{table}[H]\centering
        \begin{tabular}{c|ccccl}
        工人$A$ & 49 & 52 & 53 & 47 & 50 \\ \hline
        工人$B$ & 56 & 48 & 58 & 46 & 55
        \end{tabular}
    \end{table}
    \renewcommand{\arraystretch}{1.0}
    设两样本独立且数据所属的两总体的概率密度至多差一个平移.问能否认为工人$A$、工人$B$
    平均每天完成的件数没有显著差异($\alpha=0.1$)?






    \item \begin{enumerate}
        \item 设总体服从$N(\mu, 100)$,$\mu$未知,现有样本:$n=16,\overline{x}=13.5$,试检验假设$H_0:\mu\leq 10,H_1:\mu>10$,
        (\romannumeral1)取$\alpha=0.05$,(\romannumeral2)取$\alpha=0.10$,(\romannumeral3)$H_0$可被拒绝的最小显著性水平.
        \item 考察生长在老鼠身上的肿块的大小.以$X$表示在老鼠身上生长了15天的肿块的直
        径(以mm计),设$X\sim N(\mu,\sigma^2)$,$\mu,\sigma$均未知.今随机地取9只老鼠(在它们身上的肿块都长
        了15天),测得$\overline{x}=4.3,s=1.2$,试取$\alpha=0.05$,用$p$值检验法检验假设$H_0:\mu=4.0,H_1:\mu\neq 4.0$,
        求出$p$值.
        \item 用$p$值检验法检验$\S$2例4的检验问题.
        \item 用$p$值检验法检验第27题中的检验问题.
        

    \end{enumerate}




  

\end{enumerate}
\end{document}