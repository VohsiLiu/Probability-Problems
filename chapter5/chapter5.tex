\documentclass[10pt,a4paper]{article}
\usepackage[UTF8]{ctex}
\usepackage{fontspec}
\usepackage{geometry} 
\usepackage{amsmath}
\usepackage[shortlabels]{enumitem}
\usepackage{float}
\usepackage{graphicx}
\usepackage{subfigure}
\usepackage{epstopdf}
\usepackage{amsmath,amssymb}
\usepackage{diagbox}
\usepackage{setspace}
\usepackage{enumitem}
\DeclareSymbolFont{EulerExtension}{U}{euex}{m}{n}
\DeclareMathSymbol{\euintop}{\mathop} {EulerExtension}{"52}
\DeclareMathSymbol{\euointop}{\mathop} {EulerExtension}{"48}
\let\intop\euintop
\let\ointop\euointop

\geometry{left=3.17cm,right=3.17cm,top=2.53cm,bottom=2.54cm}
%\setmainfont{Times New Roman}
\pagestyle{plain}
\setlist[enumerate,1]{label=\textbf{\arabic*.}}
\setlist[enumerate,2]{label=(\arabic*)}

\begin{document}

\begin{enumerate}

    
    % \vspace{-0.5cm}
    % \begin{spacing}{2.0}
    % $$f(x)\left\{\begin{array}[]{ll}
        
    %     \dfrac{1}{1500^2}x, & 0\leq x \leq 1500,\\
    %     -\dfrac{1}{1500^2}(x-300), & 1500<x\leq 3000,\\
    %     0, &  \mbox{其他}.
    % \end{array}\right.$$
    % \end{spacing}
    % \vspace{-0.5cm}
    



    \item 据以往经验,某种电器元件的寿命服从均值为100$\, $h的指数分布.现随机地取16只,
    设它们的寿命是相互独立的,求这16只元件的寿命的总和大于1920$\, $h的概率.
    \clearpage


    \item \begin{enumerate}
        \item 一保险公司有10000个汽车投保人,每个投保人索赔金额的数学期望为280美
        元,标准差为800美元.求索赔总金额超过2700000美元的概率.
        \item 一公司有50张签约保险单.各张保险单的索赔金额为$X_i,i=1,2,\cdots,50$(以千美元
        计) 服从韦布尔分布,均值$E(X_i)=5$,方差$D(X_i)=6$,求50张保险单索赔的合计
        金额大于300的概率(设各保险单索赔金额是相互独立的).
    \end{enumerate}
    \clearpage


    \item 计算器在进行加法时,将每个加数舍入最靠近它的整数,设所有舍入误差相互独立且
    在$(-0.5,0.5)$上服从均匀分布
    \begin{enumerate}
        \item 将1500个数相加,问误差总和的绝对值超过15的概率是多少?
        \item 最多可有几个数相加使得误差总和的绝对值小于10的概率不小于0.90?
    \end{enumerate}
    \clearpage



    \item 设各零件的重量都是随机变量,它们相互独立,且服从相同的分布,其数学期望为
    0.5$\, $kg,均方差为0.1$\, $kg,问5000只零件的总重量超过2510$\, $kg的概率是多少?
    \clearpage



    \item 有一批建筑房屋用的木柱,其中80\%的长度不小于3$\, $m,现从这批木柱中随机地取
    100根,求其中至少有30根短于3$\, $m的概率.
    \clearpage


    \item 一工人修理一台机器需两个阶段,第一阶段所需时间(小时)服从均值为0.2的指数
    分布,第二阶段所需时间服从均值为0.3的指数分布,且与第一阶段独立。现有20台机器需要修理,求
    他在8h内完成的概率.
    \clearpage



    \item 一食品店有三种蛋糕出售,由于售出哪一种蛋糕是随机的.因而售出一只蛋糕的价格是
    一个随机变量,它取1元、1.2元、1.5元各个值的概率分别为0.3、0.2、0.5. 若售出300只蛋糕.
    \begin{enumerate}
        \item 求收入至少400元的概率.
        \item 求售出价格为1.2元的蛋糕多于60只的概率.
    \end{enumerate}
    \clearpage



    \item 一复杂的系统由100个相互独立起作用的部件所组成,在整个运行期间每个部件损
    坏的概率为0.10.为了使整个系统起作用,至少必须有85个部件正常工作,求整个系统起作
    用的概率.
    \clearpage



    \item 已知在某十字路口,一周事故发生数的数学期望为2.2.标准差为1.4
    \begin{enumerate}
        \item 以$\overline{X}$表示一年(以52周计)此十字路口事故发生数的算术平均,试用中心极限定理
        求$\overline{X}$的近似分布,并求$P\{\overline{X}<2\}$.
        \item 求一年事故发生数小于100的概率.
    \end{enumerate}
    \clearpage





    \item 某种小汽车氧化氮的排放量的数学期望为0.9$\, $g/km,标准差为1.9$\, $g/km.某汽车公
    司有这种小汽车100辆.以$\overline{X}$表示这些车辆氧化氮排放量的算术平均,问当$L$为何值时$\overline{X}>L$
    的概率不超过0.01.
    \clearpage



    \item 随机地选取两组学生,每组80人,分别在两个实验室里测量某种化合物的pH值。各人
    测量的结果是随机变量,它们相互独立,服从同一分布,数学期望为5,方差为0.3.以$\overline{X},\overline{Y}$分
    别表示第一组和第二组所得结果的算术平均.
    \begin{enumerate}
        \item 求$P\{4.9<\overline{X}<5.1\}$.
        \item 求$P\{-0.1<\overline{X}-\overline{Y}<0.1\}$
    \end{enumerate}
    \clearpage



    \item 一公寓有200户住户.一户住户拥有汽车辆数$X$的分布律为
    \begin{table}[H]\centering
    \begin{tabular}{c|ccc}
    $X$   & 0 & 1   & 2   \\ \hline
    $p_k$ & 0.1  & 0.6 & 0.3
    \end{tabular}
    \end{table}
    \vspace{-0.5cm}
    问需要多少车位,才能使每辆汽车都具有一个车位的概率至少为0.95.
    \clearpage



    \item 某种电子器件的寿命(小时)具有数学期望$\mu$(未知),方差$\sigma^2=400$.为了估计$\mu$,随机
    地取$n$只这种器件,在时刻$t=0$投入测试(测试是相互独立的)直到失效,测得其寿命为$X_1,X_2,\cdots,X_n$,
    以$\displaystyle{\overline{X}=\frac{1}{n}\sum_{i=1}^n X_i}$作为$\mu$的估计,为使
    $P\{|\overline{X}-\mu|<1\}\geq 0.95$,问$n$至少为多少?
    \clearpage




    \item 某药厂断言,该厂生产的某种药品对于医治一种疑难血液病的治愈率为0.8 .医院任
    意抽查100个服用此药品的病人,若其中多于75人治愈,就接受此断言.否则就拒绝此断言.
    \begin{enumerate}
        \item 若实际上此药品对这种疾病的治愈率是0.8,问接受这一断言的概率是多少?
        \item 若实际上此药品对这种疾病的治愈率为0.7,问接受这一断言的概率是多少?
    \end{enumerate}

    

    


  

\end{enumerate}
\end{document}